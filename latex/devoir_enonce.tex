\documentclass[12pt, letterpaper]{article}

%-----------------------------------------
% Frame du document
%-----------------------------------------
\usepackage[margin=2.5cm]{geometry}  % Définit les dimensions des marges
\usepackage{amsmath, amssymb}
\usepackage[french]{babel}
\usepackage[utf8]{inputenc}
\usepackage[T1]{fontenc}
\usepackage{helvet}

%-----------------------------------------
% Core
%-----------------------------------------
\usepackage{fancyhdr}           % Numérotation des pages, headers
\fancyhf{}
\usepackage{hyperref}           % Références hypertexte
\usepackage{booktabs,multirow,hhline}        % Tableaux
\usepackage{graphicx}           % Figures
\usepackage{subfig,wrapfig,caption}        % Sous-figures, ancrées, mise en forme des captions
\usepackage{titlesec}            % Mise en forme des sections
\usepackage{enumitem}           % Personnaliser les énumérations
\usepackage{color}            % Texte de couleur
\usepackage[dvipsnames]{xcolor}         % Plus de couleurs funky
\usepackage{textcomp}
\usepackage{lastpage}

%-----------------------------------------
% Math
%-----------------------------------------
\usepackage{amsmath,amssymb,amsthm,nicefrac}      % Symboles et versatilité mathématique
\usepackage{mathrsfs}           % ¯\_(ツ)_/¯ Polices d'écriture en math
\usepackage{wasysym,marvosym}         % Autres symboles math
\usepackage{mathtools}           % Peaufine la configuration des équations (when used)
\usepackage{dsfont}

%-----------------------------------------
% Physique
%-----------------------------------------
\usepackage{tikz}            % Dessine des figures
%\usepackage[american]{circuitikz}         % Dessine des schémas de circuits électroniques
\usetikzlibrary{quantikz}

\usepackage{verbatim}           % Je l'utilise pour écrire en verbatim et pour les commentaires
%\usepackage{minted}           % Ajoute du langage de prog élégamment
\usepackage{lipsum}            % Génère le lorem ispum
\usepackage{siunitx}            % Unités du système international avec \si
\usepackage{cancel}
\usepackage{todonotes}
\usepackage{empheq}
\usepackage{physics}
% \usepackage{bm}% Your new best friend in LaTeX
% http://ctan.math.ca/tex-archive/macros/latex/contrib/physics/physics.pdf      La documentation dudit merveilleux package

\usepackage{float}

\usepackage{halloweenmath} % décorations d'halloween pour vos devoirs (J'encourage l'utilisation de la sorcière mathémagique avec \mathwitch )

\def\CQFD{\begin{flushright}CQFD.\end{flushright}}
\def\RANCHITUP{\begin{flushright}CQFD.\end{flushright}}
\def\PIFPAF{\begin{flushright}CQFD.\end{flushright}}
% \RANCHITUP ou \PIFPAF pour écrire un CQFD bien placé

\newcommand{\uvec}[1]{\boldsymbol{\hat{\textbf{#1}}}}
\newcommand{\uveci}{{\bm{\hat{\textnormal{\bfseries\i}}}}}
\newcommand{\uvecj}{{\bm{\hat{\textnormal{\bfseries\j}}}}}
% Beaux vecteurs unitaires avec \uvec


\newcommand{\del}[2]{\frac{\partial #1}{\partial #2}}
\newcommand{\delp}[1]{\frac{\partial }{\partial #1}}
\newcommand{\ddfrac}[2]{\frac{\dd #1}{\dd #2}}
\newcommand{\ddfracp}[1]{\frac{\dd }{\dd #1}}
\newcommand{\braAket}[3]{\left<#1\left|#2\right|#3\right>}


%-----------------------------------------
% Références
%-----------------------------------------
\numberwithin{table}{section}
\numberwithin{figure}{section}
\numberwithin{equation}{section}


%|------------- Bibliography ---------------------------------|
\usepackage[style=ieee]{biblatex}
\usepackage{fmultico}
\addbibresource{references.bib}
%|------------------------------------------------------------|


\begin{document}

\title{Devoir 5 : Réseaux de Hopfield}
\author{PHQ404}
\date{Date de complétion: 19 avril 2024 à 23h45}
\maketitle

\section{Objectif}\label{sec:objectif}

\noindent L'objectif de ce devoir est d'implémenter un réseau de Hopfield.


\section{Introduction}\label{sec:introduction}
\noindent John Hopfield (1933- ) est un physicien et biologiste théorique américain dont les contributions ont
profondément influencé le domaine de l'intelligence artificielle et de la neurosciences computationnelles.
Il est particulièrement célèbre pour avoir introduit en 1982 le réseau de Hopfield, un type de réseau de neurones
artificiels avec une capacité de mémoire associative, capable de servir comme système de récupération de l'information
avec correction d'erreur.
Le modèle de Hopfield, utilisant une dynamique énergétique pour atteindre des états stables qui représentent la
mémoire.
Ce modèle a fourni un mécanisme pour comprendre comment les réseaux de neurones peuvent auto-organiser l'information
et a inspiré de nombreuses recherches sur les algorithmes d'optimisation et les modèles de mémoire dans le cerveau.
Les travaux de Hopfield ont également contribué à établir des ponts entre la physique et la biologie computationnelle,
enrichissant la compréhension des systèmes complexes et du calcul neuronal.
Sa vision interdisciplinaire a jeté les bases de nombreuses avancées dans l'étude des systèmes complexes,
tant biologiques qu'artificiels.

\bigskip

\noindent Articles fondateurs:
\begin{itemize}[label=\textbullet]
    \item \fullcite{hopfield1982neural}~\cite{hopfield1982neural}
    \item \fullcite{hopfield1984neurons}~\cite{hopfield1984neurons}
\end{itemize}

\noindent Le réseau de Hopfield est un réseau de neurones de type McCullogh et Pitts tel que:
\begin{itemize}[label=\textbullet]
    \item Le réseau est \textbf{complet et simple}: chaque neurone est connecté à tous les autres neurones sauf à
    lui-même. Il est donc récurrent mais sans boucle.
    \item De plus, les poids de connexions sont \textbf{symétriques}, c'est-à-dire $W=W^\top$ et donc $w_{ij}=w_{ji}$
    pour tous les $i,j$.
    \item Les poids de connexion peuvent être négatifs, ce qui signifie que le réseau peut avoir de
    l'\textbf{inhibition}, c'est-à-dire le processus neurologique par lequel une cellule nerveuse (neurone)
    réduit l'activité d'une autre cellule nerveuse, limitant ainsi ou empêchant la transmission de signaux excitateurs.
    \item Les poids de connexion sont plastiques et changent selon une \textbf{règle de Hebb}, souvent résumée
    par l'adage "ce qui se renforce ensemble, se connecte ensemble", est un principe fondamental en neurosciences
    et en théorie des réseaux de neurones artificiels qui décrit comment les connexions entre les neurones du
    cerveau se renforcent durant l'apprentissage.
    Formulée en 1949 par \textbf{Donald Hebb} (1904-1985), elle stipule que si deux neurones sur les deux
    côtés d'une synapse (connexion) sont activés simultanément, alors la force de cette connexion augmente.
    En d'autres termes, la répétition de l'activation simultanée renforce la tendance de ces neurones à
    s'activer ensemble à l'avenir.
\end{itemize}


\subsection{Dynamique d'un réseau de Hopfield}\label{subsec:dynamique-d'un-reseau-de-hopfield}
\begin{itemize}[label=\textbullet]
    \item \textbf{Activité binaire} : l'activité du neurone $i$ à l'instant $t$, $x_i(t)$, est égale à 0 ou 1.
    \item \textbf{Temps discret} : $t$ appartient à un ensemble dénombrable (discret) et pour simplifier on suppose
    que $t\in\{0,1,2,\ldots\}$.
    \item \textbf{Mise à jour asynchrone} : À tout instant $t$, l'activité d'un et d'un seul neurone peut être modifiée.
    L'indice du neurone mis à jour est choisi aléatoirement à tout instant $t$.
    \item \textbf{Règle de mise à jour} : Si le neurone $i$ à l'instant $t$ est mis à jour , alors
    \[
    x_i(t+1) = H\left(\sum_{j=1}^nw_{ij}x_j(t)-\theta_i\right),
    \]
    ce qui signifie que
    \[
    x_i(t+1) =
    \begin{cases}
    1 & \text{si }\sum_{j=1}^nw_{ij}x_j(t)>\theta_i, \\
    0 & \text{si }\sum_{j=1}^nw_{ij}x_j(t)\leq\theta_i .
    \end{cases}
    \]
    Sous forme vectorielle,
    \[
    \mathbf{x}(t+1) = H\left(W \mathbf{x}(t)-\boldsymbol{\theta}\right).
    \]
\end{itemize}

\subsection{Quelques concepts fondamentaux}
\begin{itemize}[label=\textbullet]
    \item \textbf{État stationnaire} : si $\mathbf{x}(t)= \mathbf{x}^*$ pour tout $t$, alors $\mathbf{x}^*$ est un
    état stationnaire.
    \item \textbf{Fonction d'énergie} (Lyapunov) :
    \[
    E(\mathbf{x}) = -\frac{1}{2}\sum_{i\neq j} w_{ij}x_ix_j +\sum_i\theta_ix_i=-\frac{1}{2}\mathbf{x}^\top W \mathbf{x} + \boldsymbol{\theta}^\top \mathbf{x}.
    \]
    \item \textbf{Résultat important de Hopfield} : avec la mise à jour asynchrone et les connexions symétriques,
    \[
    E\big(\mathbf{x}(t+1)\big) \leq E\big(\mathbf{x}(t)\big),
    \]
    ce qui signifie que les minima locaux de $E$ sont des états stationnaires stables.
    \item \textbf{Mémoire} : un état stationnaire est un état mémorisé.
    En partant d'un état proche, on peut converger au fil du temps vers cet état ou vers un autre état stationnaire.
\end{itemize}

\subsection{Plasticité du réseau de Hopfield : apprentissage d'états}
\begin{itemize}[label=\textbullet]
    \item \textbf{Apprentissage d'un état} : pour stocker un seul état $\mathbf{y}$, on change $W_{i,j}$ pour
    \[
    W_{ij} +
    \begin{cases}
    (2y_i - 1)(2y_j - 1)&  \text{si} \quad i \neq j, \\
    0 &  \text{si} \quad i= j.
    \end{cases}
    \]
    \item \textbf{Apprentissage de plusieurs états} : pour stocker les $\mathbf{y}^\mu$, $\mu=1,\ldots, p$,  on change $W_{i,j}$ pour
    \[
    W_{ij} +
    \begin{cases}
    \sum_{\mu}(2y^\mu_i - 1)(2y^\mu_j - 1)&  \text{si} \quad i \neq j, \\
    0 &  \text{si} \quad i= j.
    \end{cases}
    \]
    \item \textbf{Lien avec Hebb} : les termes $y^\mu_i y^\mu_j $ ajoutent du poids à la connexion $(i,j)$ si
    les neurones $i$ et $j$ sont activés quand le réseau est dans l'état $\mathbf{y}^\mu$.
\end{itemize}



\section{Comment présenter et remettre votre TP}\label{sec:comment-presenter-et-remettre-votre-tp}

\noindent Vous devez cloner le répertoire github dans l'organisation du cours au lien suivant :\\
\href{https://classroom.github.com/a/FJR5qORQ}{https://classroom.github.com/a/FJR5qORQ}.
Dans ce répertoire se trouvera votre code python, vos tests unitaires ainsi que votre rapport
décrivant les méthodes utilisés et l'analyse de vos résultats.
La structure des fichiers ne doit pas être modifiée, mais vous pouvez ajouter des fichiers si vous le désirez.
Voici la structure de fichiers que votre répertoire devra garder :

\bigskip

Root
\begin{itemize}
    \item[]
        \begin{itemize}
            \item[$\rightarrow$] src
                \begin{itemize}
                    \item[$\hookrightarrow$] \texttt{fichier0.py}
                    \item[$\hookrightarrow$] \texttt{fichier1.py}
                    \item[$\hookrightarrow$] \dots
              \end{itemize}
        \end{itemize}
  \item[]
  \begin{itemize}
    \item[$\rightarrow$] tests
    \begin{itemize}
      \item[$\hookrightarrow$] \texttt{test\_fichier0.py}
      \item[$\hookrightarrow$] \texttt{test\_fichier1.py}
      \item[$\hookrightarrow$] \dots
    \end{itemize}
  \end{itemize}
  \item[$\hookrightarrow$] \texttt{.gitignore}
  \item[$\hookrightarrow$] \texttt{requirements.txt}
  \item[$\hookrightarrow$] \texttt{README.md}
\end{itemize}

\bigskip

\noindent Le fichier \texttt{requirements.txt} doit contenir les dépendances de votre projet.
Le fichier \\\texttt{README.md} doit contenir les instructions pour installer et utiliser votre projet ainsi
qu'une brève description du devoir et des méthodes utilisés dans le code.
Voir la section~\ref{sec:Readme} pour plus de détails.
Dans le dossier \texttt{src} se trouvera votre code python et dans le dossier \texttt{tests} se trouvera vos tests
unitaires.

\bigskip

\noindent La remise et la correction automatique du code se fera à chaque \texttt{push} sur le répertoire github.
Notez que seul le dernier \texttt{push} sur la branche \texttt{main} sera considéré pour la correction.


\section{Énoncé}\label{sec:enonce}

\subsection{Modèle de Hopfield}\label{subsec:modele}

\noindent Vous allez devoir implémenter une classe \texttt{HopfieldNetwork} qui permet de créer un réseau de Hopfield.
Dans celle-ci, vous devrez implémenter les méthodes suivantes:
\begin{itemize}[label=\textbullet]
    \item \texttt{\_\_init\_\_} : le constructeur de la classe qui prend en argument les poids de connexions
    \texttt{weights}, les seuils \texttt{thresholds} et l'état initial \texttt{initial\_state} du réseau.
    \item \texttt{n} : Une propriété qui retourne le nombre de neurones dans le réseau.
    \item \texttt{energy} : Une méthode qui retourne l'énergie du réseau.
    \item \texttt{get\_state\_energy} : Une méthode qui calcule l'énergie d'un état donné.
    \item \texttt{set\_state} : Une méthode qui permet de changer l'état du réseau.
    \item \texttt{update} : Une méthode qui met à jour l'état du réseau de manière asynchrone sur un pas de temps.
    \item \texttt{simulate} : Une méthode qui met à jour l'état du réseau de manière asynchrone sur un nombre de
    pas de temps \texttt{m} et qui retourne l'historique des états du réseau.
    \item \texttt{learn\_one\_state} : Une méthode qui permet d'apprendre un état \texttt{state} en modifiant les
    poids de connexions.
    \item \texttt{learn\_many\_states} : Une méthode qui permet d'apprendre plusieurs états \texttt{states} en modifiant les
    poids de connexions.
\end{itemize}

\subsection{Visualisation des résultats}\label{subsec:visualisation-des-resultats}

\noindent Vous allez devoir présenter les graphiques suivants:
\begin{itemize}[label=\textbullet]
    \item L'énergie du réseau en fonction du temps pour des états mémorisés différents.
    \item L'évolution de l'état du réseau en fonction du temps pour des états mémorisés différents.
    (Utilisez des gifs)
\end{itemize}

\noindent Vous allez devoir faire ces derniers graphiques pour les états `image` et `message` qui sont dans le
dossier \texttt{tests/data}.
En plus, vous allez devoir ajouter une petite analyse sur l'apprentissage de plusieurs états
(fait avec la méthode \texttt{learn\_many\_states}) et sur la dynamique du réseau pour ces états.


\bigskip

\noindent Note: Toutes les fonctions qui doivent être implémentées sont déjà définies dans les fichiers
et retournent des \texttt{NotImplementedError}.


\section{Vérification}\label{sec:verification}

\noindent Il est important de vérifer vos implémentations.
En effet, vous devez vous assurer que vos méthodes fonctionnent correctement et pour ce faire, vous devez rouler et
implémenter des tests unitaires qui testent chacune de vos classes et fonctions.
De plus, vous devriez tester si les résultats obtenus sont logiques.
Il serait aussi intéressant de retrouver vos vérifications dans votre rapport.
Il est fortement recommandé d'ajouter des tests unitaires dans le dossier \texttt{tests}, mais les tests déjà
implémentés ne doivent pas être modifiés.


\section{Readme}\label{sec:Readme}

\noindent Vous devez faire un fichier Readme qui explique ce que contient votre répertoire et comment l'utiliser.
Le Readme sera divisé en 2 parties: une partie plus courte qui consiste essentiellement à ce qu'on retrouve normalement
dans Readme scientifique et une partie plus longue qui consiste en une présentation et analyse des résultats.

\bigskip

\noindent La première partie doit contenir les éléments suivants:
\begin{itemize}
    \item Une brève description du contenue du répertoire;
    \item Une figure qui résume le contenu du répertoire ainsi que les résultats principaux;
    \item Les instructions pour installer et utiliser votre projet.
\end{itemize}
Il faut qu'un utilisateur externe soit en mesure de regarder la première partie du Readme et comprendre en quelques
secondes le contenu du répertoire, les résultats principaux et comment utiliser le projet.
C'est important d'être concis, clair et efficace.
Pour la figure, il sagit d'une image permettant au lecteur de comprendre rapidement le contenu du répertoire.
Celle-ci pourrait être, par exemple, un diagramme représentant le pipeline de traitement des données, un graphique
comparant les différentes méthodes implémentées, etc.

\bigskip

\noindent Le deuxième partie doit contenir les éléments suivants:
\begin{itemize}
    \item Une plus longue description du contenu du répertoire;
    \item Une présentation et explication des méthodes utilisées;
    \item Une présentation des résultats obtenus;
    \item Une analyse des résultats obtenus;
    \item Une conclusion.
\end{itemize}
Cette seconde partie sert à expliquer en quoi consiste ce dépôt si l'utilisateur décidait que la première partie
du Readme était assez intéressante et bien présentée pour qu'il veuille en savoir plus.
Il sagit ici d'un court rapport scientifique.
Il faut donc rester concis afin d'être lu en quelques minutes seulement, mais mettre suffisamment d'information pour
que l'utilisateur comprenne bien la théorie, les méthodes et les résultats.



\section{Critères d'évaluation}\label{sec:criteres-d'evaluation}

\begin{description}
  \item[70 points] Pour le résultat de l'autocorrection du code obtenue à l'aide du module TAC\@.
  \item[30 points] Pour la qualité du Readme.
\end{description}


\newpage
\printbibliography

\end{document}
